\clearpage
\newpage
\setcounter{page}{2}
\renewcommand\thepage{\arabic{page}}
\fancyhead{}
\fancyhead[C]{\large(\thepage)}

\hangindent=2ex
{
    \itshape
    \bfseries
    \noindent
    \marginpar{\raggedleft\textbf{Epistle.}}
    Mr.\ \emph{Lock}'s= Letter to \emph{Monsieur Toinard},
    containing a New and Easie Method of making \emph{Common-Place Books=},
    an exact Index of which may be made in Two Pages=.
}

\savebox{\dropcap}{\scalefont{2.8}I}
\settoheight{\dropcapheight}{\usebox{\dropcap}}
\settowidth{\dropcapwidth}{\usebox{\dropcap}}
\smash{\raisebox%
    {-\dimexpr \dropcapheight+0.6\thefontsize\relax}%
    {\noindent\usebox{\dropcap}}}%

\setlength{\shapeindent}{\dimexpr \dropcapwidth+1em\relax}
\setlength{\shapelength}{\textwidth}
\addtolength{\shapelength}{-\shapeindent}
\parshape=3
\shapeindent \shapelength
\shapeindent \shapelength
0pt \textwidth
\noindent
\begin{linenumbers}
    Do at length, Sir, obey you in Publishing my Method of making \emph{Common-Place Books=}.
    I am ashamed that I should be so backward in Complying with your Desires=;
    but that which you requested of me,
    seemed to me a Thing so inconsiderable,
    that I thought it not worthy of publick View,
    especially in an Age so abounding with Fine Inventions= as= ours=.

    You know that I voluntarily communicated this= Method to you,
    as= I have done to many others=,
    to whom I believ'd it would not be unacceptable.
    It was= not then,
    as= if I design'd it for my own private Use alone,
    that I have hitherto refus='d the making of it Publick.
    I was= of Opinion, that the Respect which One ought to have for the Publick,
    would not suffer me to present it with an Invention of so small an Importance;
    but the Obligations= which you have laid me under,
    and our Common Friendship,
    do not permit me any longer to decline the following your Advice.
    Your last, Sir, has= wholly determined me,
    and I believe that I ought not to stick at the Publishing my Method since you tell me that you have found it very useful after a Trial of several Years=,
    as= well as= those of your Friends=,
    to whom you have Communicated it.
    It is= needless= for me here to relate what Profit I my self have reap'd by the Use of it for above Twenty Years=.

    I have sufficiently Entertained you with it when I was= at \emph{Paris=},
    about Seven or Eight Years= ago,
    while I might have receiv'd great Benefit by your Learned and Agreeable Conversation.
    All the Advantage that I aim at from this= Writing is= publickly to testifie the Esteem and Respect that I have for you,
    and to declare how much I am, Sir, Your, \&c.
\end{linenumbers}

Before we come to the Matter in Hand, it may not be amiss= to remark that this= Method is= put in the same Order that the Collections= ought to be put in.
you will perceive by the Reading of that which follows= what the Heads= mean, which you see at the Top of the Back of every Leaf, and at the Bottom of this= Page.

\place{Ebionites}
In the Gospel of the Ebionites=, which they called the Gospel according to the Hebrews=, the History which is= in Matthew XIX.
16.
and the following ones=, was= thus= alter'd;
One of the Rich men said unto him, Master, What good thing must I do that I may have Eternal Life?
Jesus= said unto him, obey the Law and the Prophets=.
He answered, I have done it.

Jesus= said unto him, go sell

Adversariorum Methodus=, or the Method of Common Places=.

I take a White Paper Book of what Size I think fit, I divide the Two First Pages= which face one another, by parallel Lines=, into Five and Twenty equal parts=, with Black Lead;
after that I cut them perpendicularly by other Lines=, which I draw from the Top of the Page to the Bottom, as= you may see in the Table or Index, which I have put before this= Writing.
Afterwards= I mark with Ink every Fifth line of the Twenty Five that I just now spoke of.

[The other Lines= are made with Red Lead, but for Conveniency one may make them with Black Lead, which is= better for Use than Red Lead.]

I put at the Beginning of every Fifth Space, or before the Middle, One of the Twenty Letters= which are design'd for this= Use;
and a little farther in every Space, One of the Vowels= in their Natural Order.
This= is= the Index or Table of the whole Volume, be it of what Size soever.

The Index being thus= made, I mark out, in the other Pages= of the Book, the Margin with Black Lead;
I make it about the bigness= of an Inch, or a little bigger, if the Volume be in folio, but in a less= Volume the Margin is= proportionably less= also.

If I would put any Thing in my Common Place Book, I look a Head to which I may refer it, that I may be able to find it when I have Occasion.
Every Head ought to begin with some Considerable Word that is= Essential to the Matter treated of, and of this= Word one must carefully observe the First Letter, and the Vowel which follows= it;
for upon these Two Letters= depends= the whole Use of our Index.

I leave out Three Letters= of the Alphabet as= useless=, to wit, K, Y, W, which are supplied by C, I, U, Letters= of a like Power.
I put the Letter Q, which is= always= followed by an U in the Fifth Space of Z.
By this= throwing of the Letter Q into the last Space of my Index, I preserve its= Uniformity, and do not at all shorten the Length of it:
For it very seldom happens= that one meets= with an Head that begins= with Z U, and I have not found so much as= One for the Space of Five and Twenty Years=, during which Time I have made use of this= Method.
Yet nevertheless=, if it be needful, nothing will hinder but that you may put it in the same Space with Q U, provided you make some Sort of Distinction.

But one may, for more Exactness=, assign to Q a Place at the Bottom of the Index, which I have done formerly.
When I meet with any thing worth putting into my Common-Place-Book, I presently look for a proper Head.
Suppose, for Example, the Head were Epistle;
I look in the Index the First Letter with the Vowel that follows=, which in this= Case are E I.
If there is= found any Number in the Space marked E I, that shows= me the Space design'd for Words= which begin with E, and whose Vowel that immediately follows= is= I, I must refer to the Word Epistle in that Page what I have to take notice of.
I Write the Head in pretty large Letters=, so that the principal Word is= found in the Margin, and I continue the Line in writing on what I have to remark.
I constantly observe this= Method, that nought but the Head appear in the Margin, and go on without carrying the Line again into the Margin.
When one has= thus= preserv'd the Margin clear, the Heads= present themselves= at First Sight

If in the Index I find no Number in the Space E I, I look in my Book the First Backside of the Page that I find blank, which Backside in a Book where there is= nothing else but the Index, must be the Second Page.

I write then in my Index after E I the Number 2, and the Head \textsc{Epistle} at the Top of the Margin of the Second Page, and all that is= to be put under this= Head in the same Page, as= you see I have done in the Second Page of this= Writing.

Since the Class= E I.
does= solely take up the Second and Third Page, one may make Use of those Pages= only for Words= which begin, with E, and whose next Vowel is= I, as= Epicurus=, Ebionites= [1], Epigram, Edict, \&c.

[1]
See the Bottom of the Third Page.

The Reason why I always= begin at the Top of the Back of the Page, and that I assign to one Class= the Two Pages= which face one another, rather than a whole Leaf, is= that the Heads= of this= Class= may appear all at once, otherwise you must be at the Trouble of turning over the Leaf.

Whensoever I would write a new Head, in my Common-Place-Book, I look presently in my Index for the Characteristick Letters= of the Word, and I see by the Number which follows= them where the Page assign'd to the Class= of this= Head is=.
But if there be no Number to be found, I must look the First Back of the next Blank Page.
I mark its= Number in the Index, and so I preserve this= Page, with the Right Side of the following Leaf for this= new Class=.
Let the Word be for Example Adversaria, if I see no Number in the Space A E, I look the First Empty Back of a Leaf, which finding in the Fourth Page, I put in the Space A E the Number 4, and in the 4th Page the Head Adversaria, with all that ought to be put under this= Head, as= I have already done.

After that, this= Fourth Page with the Fifth that follows= it, is= reserv'd for the Class= A E;
that is= to say for Heads= which begin with A and where the following Vowel in the Word is= E, as= Agesilaus=, Acheron, Anger, \&c.

When the Two Pages= design'd for this= Head are quite full, then look for the Back of the next Blank Page:
If it is= that which immediately follows=, I write at the Bottom of the Margin in the Page that I have last filled up, the Letter V, that is= Verte, Turn over;
and likewise at the Top of the Page following.
If the Pages= which immediately follow are already taken up by other Classes=, I write at the Bottom of the Page last filled up the Number of the next Back of the Page.
I set down again the Head of which it treats= under which I go on to write what I have to put into my CommonPlace-Book, as= if it were in the same Page.
At the Top of this= new Back I set down also the Number of the Page which has= been last filled up.
By these Numbers=, which refer to one another, the 1st of which is= at the End of one Page, and the 2d at the Beginning of another, one reads= the Matter which is= separated as= well as= if there was= nothing between them.
For by this= Reciprocal Reference of Numbers=, one turns= over as= one Leaf all those Which are between them, as= if they were join'd together.
You have an Example of it in the Third and Fourteenth Pages=.

Every Time I put a Number at the Bottom of a Page, I put it also in the Index;
but when I put only V, I make no Alteration in the Index;
the Reason of which you will learn by Use.

If the principal Word of the Head be a Monosyllable, (or a Word of One Syllable) and begins= with a Vowel, this= Vowel is= at the same Time both the First Letter of the Word, and the Characteristick Vowel;
so I write the Word Art in A a, and Elf in E e.

It may be seen by what I have said, that one is= to begin to write every Class= of Words= on the Back of the Page;
It may from thence happen that the Backs= of all the Pages= may be full, while there may be [1]
Right Sides= enough which do yet remain empty.
If you have a Mind then to fill up the Book, you may assign these Right Sides=, which are yet entirely blank, to new Classes=.

[1]
See the Fifteenth, Seventeenth and Nineteenth Pages=, \&c.

If any one thinks= that these Hundred Classes= are not sufficient to take in all Sorts= of Subjects= without Confusion, he may, following the same Method, increase the Number to Five Hundred, by adding a Vowel.
But having try'd both the one and the other Method;
I prefer the former, and Use will convince those that shall make Tryal of it that it is= sufficient for all Subjects=, especially if they have a Book for every Science, in which they make their Collections=, or at least Two, for the Two Heads= to which we may refer all our Knowledge, to wit, Moral and Natural Philosophy.

One may also add to them a Third Book, which you may call the Science, or Knowledge of Signs=,which respects= the Use of Words=, and is= of far larger Extent than the Ordinary Critical Art.

As= for the Language in which one ought to express= the Titles=, I believe the Latin Tongue to be the most Convenient, provided one always= observes= the Nominative Case, least in Dissyllables=, (or Words= of Two Syllables=) or in Monosyllables= which begin with a Vowel, the Change, which happens= in the Oblique Cases= should cause Confusion.
But it does= not much matter what Language you make Use of, provided you do not mix the Heads= of different Languages= together.

To remark a Place in an Author from whom I would collect any Thing I make use of this= Method:
Before I write down any thing I put the Name of my Author in my Common-PlaceBook, and under that Name, the Title of the Treatise I am reading, the Volume, the Time and Place of the Edition, and (what ought never to be omitted) the Number of the Pages= that the whole Book contains=.
For Example, I put in the Class= M.
A.
Marshami Canon Chronicus=, Aegyptiacus=, Graecus=, \& Disquisitiones=, Fol.
Lond.
1672, p.626.
This= Number of the Pages= serves= me for the future to mark the Particular Treatise of the Author, and the Edition that I make use of.
I have no more Need to mark the Place otherwise, than by putting in the Number of the Page from whence I have Collected what I have writ over the Number of the Pages= of the whole Volume.
You will see an Example of it in Acherusia, where the Number 259 is= over the Number 626, that is= to say, the Number of the Pages= where the Place is= that is= treated of, over the Number of the Pages= of all the Volume.
So I not only avoid the Trouble of writing Canon, \&c.
but I can also, by the Help of the Rule of Three, find the same Passage in any other Edition whatever, by looking the Number of Pages= that the Edition I have not made use of contains=;
since the Edition which I have used having 626 Pages=, hath given me 259.
I confess= one does= not always= hit upon the very Page, because of the Spaces= which may be made in different Editions=, which are not always= proportionably equal;
but nevertheless= you are never very far off of the Page;
and it is= much better to find out the Passage within some few Pages= of the Place, than to be at the Trouble of turning over the whole Book to find it;
as= you must do if the Book has= no Index, or where the Index is= not very correct

\place{Acherusia.}
Pratum, ficta mortuorum Habitatio est Locus= prope Memphim, juxta Paludem, quam vocant Acherusiam, \&c.
This= is= a Passage taken out of the First Book of Diodorus= Siculus=, the Sense of which is= this=:
The Fields=, where they feign the Habitation of the Dead to be, is= a Place not far from Memphis=, near the Marsh called Acherusia, where there are most Delightful Fields=, with Lakes= and Woods= of Lotus= and Calamus=.

It is= not without Reason then that Orpheus= says= the Dead inhabit those Places=, because it is= there that the most and greatest Funeral Solemnities= of the Aegyptians= are Celebrated;
they carry the Dead over the River Nile, and the Marsh Acherusia, and lay them in Subterraneous= Vaults=.

There are other Stories= among the Grecians= concerning the Shades= below, which are very like those Stories= which are invented at this= Day in Aegypt.
For they call the Boat which carries= over the Dead Daris=, and a Piece of Money is= given to the Waterman for his= Passage, whose Name in the Language of that Country is= called Charon.
Not far from this= Place there is= the Temple of Gloomy Hecate, also the Gates= of Cocytus= and Lethe, shut up with great Brazen Bars=;
there are also other Gates=, called the Gates= of Truth, before which stands= the Statue of Justice without an Head.
Marsham 259/626.

\begin{verbatim}
                *
               * *
            * *    *
             *    * *
               * *
                *
\end{verbatim}

\place{Ebionites.}
Sell all that thou hast, and give it to the Poor, then come and follow me:
But at that the Rich Man began to scratch his= Head, and was= not at all pleas='d with the Advice that Jesus= gave him.
And the Lord said unto him, how say you I have fulfilled the Law and the Prophets=, since it is= written in the Law, thou shalt love thy Neighbour as= thy self;
and lo there are many of thy Brethren, the Children of Abraham, who have bad Raiment, and die with Hunger, while no Help is= administred to them from you, tho' your House abounds= with all Good Things=?
And having turn'd to Simon, his= Disciple, who sat next him, Simon, thou Son of Johanna, said he, it is= easier for a Camel to go through the Eye of a Needle, than for a Rich Man to enter into the kingdom of Heaven.
Ebion alter'd this= Passage of the Gospel because he did not acknowledge Christ to be the Son of God, nor a Law-giver, but a bare Interpreter of the Law which was= given by Moses=.
Grotius= 336/1060.

\place{Hereticks.}
Nostrum igitur fuit eligere \& optare meliora, ut ad vestram correctionem aditum haberemus=.
Augustinus= Tom.
VI, Col.
116.
fol.
Basileae 1542.
contra Epist.
Manichaei, quam vocant Fundamenti.
"We believed that
" other Methods= ought to be taken, in
" Order to make you retract your Er" rors=;
Affronts= and Invectives= are by
" all Means= to be avoided, ill Usage
" and Persecution are never likely to
" succeed;
but the only Way to draw
" you is= by kind Discourses= and Exhorta" tions=, which may demonstrate our ten" der Concern for you;
according to that
" of the Scripture, a Servant of the
" Lord ought not to be Quarrelsom, but
" Gentle to all Men;
Apt to teach, Pati" ent,and with Modesty, to reprove those
" that are not like-minded.
Let those Persons= rigorously treat you, who know not how difficult it is= to come to the Knowledge of Truth, and to avoid Errors=.
Let those Persons= rigorously treat you, who know not how hard a Matter it is=, and and how seldom Effected, to cause Carnal Imaginations= to give way to Spiritual and Pious= ones=.
Let those Persons= rigorously treat you, who are not sensible of the extream Difficulties= that there are to purifie the Eye of the inward Man, in Order to make it capable of perceiving Truth, which is= the Sun of the Soul.
Let

\place{Confession of Faith.}
Periculosum nobis= admodum atque etiam miserabile est, tot nunc fides= existere, quot voluntates=;
\& tot nobis= doctrinas= effe quot mores=, \&c.
Hilarius=, p.
211.
in Lib.
ad Constantium Augustum.
Basil.
1570, Fol.
It is= a Thing both Deplorable and Dangerous=, that there are now as= many Confessions= of Faith as= there are Wills=, as= many Opinions= as= Inclinations=, and as= many Sources= of Blas= phemy as= there are Vices=, whilst we make as= many Confessions= of Faith as= we please, and Gloss= upon them as= we think fit.
And as= there is= but One God, One Lord, and One Baptism, so there is= but One Faith, which One Faith we Renounce when we make many different Confessions=;
and certainly this= Diversity is= the Cause that there is= no more true Faith to be found.
We are convinc'd, that after the Council at Nice there was= nothing, either on one Side or t'other, but writing Confessions= of Faith.
And while they contend about Words=, while they debate about Novel Questions=, while they dispute about Equivocal Terms=, while they complain of Authors=, while every Body endeavours= to advance his= own Party, while no one can agree,

\place{Hereticks.}
Let those Persons= rigorously treat you, who know not how many Sighs= and Groans= it costs= before one can attain to any small Knowledge of the Divine Being.
Finally, let those Persons= rigorously treat you, who were never seduc'd by such Errors= as= they see you have been deceiv'd by.
I pass= by that most pure Wisdom, to the Knowledge of which very few Spiritual Persons= arrive at in this= Life;
yet although they know it but in very small Measure, because they are Men, yet they know it without doubting.
For in the Catholick Church it is= not Penetration of Wit, nor Depth of Knowledge, but the Simplicity of the Faith, which makes= People sure and safe.

Barbari quippe homines= Romanae, imo potius= humanae eruditionis= expertes=, qui nihil omnino sciunt, nisi quod a Doctoribus= suis= audiunt;
quod audiunt hoc sequuntur, \&c.
Salvianus= 162/339.
The Sense of which take as= follows=.
This= Bishop speaking of the Arian Goths= and Vandals=, They are a Barbarous= People, says= he, who have not any Taste of the Roman Learning, and who are ignorant even of those Things= with which almost all the rest of Mankind are acquainted;
they know nothing but what they have learnt from their Doctors=, and mind nothing but what they have heard from them.
Whence People so ignorant as= these are, find themselves= under a Necessity of learning the Mysteries= of the Gospel, rather from the Instructions= which are given them, than from the Reading of Books=, therefore the Tradition and received Doctrine of their Masters= are the only Rule that they follow, because they know nothing but what they have taught 'em.
They are Hereticks=, but they know not that they are so.
They are so indeed in our Esteem, but they don't at all believe it;
yea, on the contrary, they reckon themselves= to be true Catholicks=, and Brand us= with the Title of Hereticks=.
They judge therefore of us= just as= we do of them.
We are persuaded with our selves= that they do Wrong to the Divine Generation, in maintaining the Son to be inferiour to the Father.
They imagine that we derogate from the Glory of the Father because we believe them to be Equal.
The Truth is= on our Side, but they pretend it is= on theirs=.
We give all due Honour to God, and they think that their Belief tends= more to the Honour of God than ours=.
They are wanting in their Duty to God, but this= they count the highest Duty of Religion;
and they make true Piety to conflict in that which we have a quite contrary Opinion of.
They are then in an Error, but yet they are Sincere;
and it proceeds= not from an Hatred, but Love of God.

\place{Hereticks.}
For they pretend that by it they do better testifie the Respect they have for God, and their Zeal for His= Glory.
Therefore although they have not a right Faith, yet they nevertheless= look upon it as= a perfect Love of God.
How these Persons= will be punished for their Errors= at the Day of Judgment the great Judge of the Universe alone knows=.
In the mean Time I believe that God exercises= his= Patience towards= them, because he sees= that their Heart is= more right than their Faith;
and that when they do deceive themselves=, it is= an Affection for Piety that is= the Cause of their Error.

\place{Confession of Faith.}
While no one can agree, while they Anathematize one another, there is= scarce any that sticks= close to Jesus= Christ.
What Change was= there in the Confession of Faith but last Year?
The First Synod of the Nicene Council ordains= that nothing should be said concerning the Homousion;
the Second orders= and ordains= that they should speak of it;
the Third excuses= the Fathers= of the Council, and pretends= that they took the Word Ousia simply;
lastly, the Fourth, instead of excusing, condemns= 'em.
As= to the Resemblance of the Son with his= Father, which is= the Confession of Faith of these unhappy Times=, they dispute whether he is= like in the Whole, or only in Part.
Behold what Fine Inquirers= these are into the Secrets= of Heaven!
In the mean while, it is= upon the Account of these Confessions= of Faith about the invisible Mysteries=, and about our Faith in God, that we thus= Calumniate one another.
We make Confessions= every Year, and also every Month;
we Repent of what we have done, we Defend those that Repent of 'em, and afterwards= Anathematize those We have defended:
So we Condemn either the Opinions= of others= in our selves=, or our own Opinions= in others=;
and in thus= Tearing one another to Pieces=, we have been the Cause of each other's= Ruin.

\textsc{Finis}
