The following Account was Writ
by the late Ingenious Mr. Ol-
denburg, Secretary of the
Royal Society,

THE, Person, to whom the foregoing
Letter doth refer, is Mr. Daniel
Whaley, (Son of Mr. ....... Whaley, late
of Northampton, and Mayor of that Town)
He was (soon after the Date of this Let-
ter) on the 21st of May 1662, present at
a Meeting of the Royal Society, (of which
the Register of that Day's Proceedings
takes particular Notice) and did in their
Presence, to their great Satisfaction, pro-
nounce distinctly enough such Words as
by the Company were proposed to him;
and though not altogether with the usual
Tone or Accent, yet so as easily to be un-
derstood: Whereupon also the said Do-
ctor was, by the same Assembly, encou-
raged to pursue what he had so ingeniously
and successfully begun. About the same
Time also (his Majesty having heard of
it, and being willing to see him) he did
the like several Times at Whitehall, in the
Presence of His Majesty, his Hignness
Prince Rupert, and divers others of the
Nobility, tho' he had then employ'd but a
small Time in acquiring this Ability. In
the Space of One Year, which was the
whole Time of his Stay with Dr. Wallis,
he had read over a great Part of the Eng-
lish Bible, and had attain'd so much Skill,
as to express himself intelligibly in ordina-
ry Affairs; to understand Letters written
to him, and to write Answers to them,
tho' not Elegantly, yet so as to be under-
stood; and in the Presence of many Fo-
reigners (who out of Curiosity have come
to see him) hath oft-times not only read
English and Latin to them, but pronoun-
ced the most difficult Words of their Lan-
guages (even Polish it selfj which they
could propose to him. Since that Time,
tho' he hath not had Opportunity of ma-
king much farther Improvement, for want
of an Instructor, yet he doth yet retain
what he had attain'd to; or wherein he
may have forgot the Niceness requisite in
the Pronunciation of some Sounds, doth
easily recover it with a little Help.

Nor is this the only Person on whom
the said Doctor hath shewed the Effect of
his Skill, but he hath since done the like
for another, (a young Gentleman of a ve-
ry good Family and a fair Estate) who
did from his Birth want his Hearing. On
this Occasion I thought it very suitable to
give Notice of a small Latin Treatise, ef
this same Author, first Published in the
Year 1653, intituled De Loquela, [of
Speech] prefixed to his Grammar of the
English Tongue, written also in Latin. In
which Treatise of Speech, (to which he
refers in this Discourse, and on Confidence
of which he durst undertake that difficult
Task) he doth very distinctly lay down
the Manner of Forming all Sounds off Let-
ters usual in Speech, as well of the Eng-
lish as of other Languages; which is, I
think, the First Book ever Published of
that Kind; (for tho' some Writers for-
merly have here and there occasionally
said something of the Formation of some
particular Lettets, yet none, that I know
of, had before him undertaken to give an
Account of all.) Whether any since him
have with more Judgment and Accurate-
ness performed the same, I will not take
upon me to determine. In his Grammar
of the English Tongue, (to which this of
[1] Speech is prefixed) he hath so briefly
and clearly given an Account of this Lan-
guage, as may be very Advantageous, not
only to Strangers, for the easie Attain-
ment thereof, but even to the English them-
selves, for the clear Discovering (which
few take Notice of) the true Genius of
their own Language.

[1] A Translation of this Treatise _Of Speech_, as
likewise of the Grammar, with some considerable and
useful Additions, is pre-
paring for the Press; the Whole will be looked over by several Learn-
ed Men, both of Town and the Universities. If any Gentlemen that
have made Observations on the English Tongue, will be pleas'd to
communicate them to the Bookseller, they shall be carefully inserted
in their proper Places.
