A Letter of Dr. John Wallis,
(Geom. Prof. Oxon, and F.
R.S.) to Mr. Thomas Be-
verly, concerning his Method
for instructing Persons Deaf
and Dumb.

SIR,

I Have receiv'd your Letter of Sept. 22.
wherein you tell me the Case of a Fa-
mily, wherein you are concern'd;
which is really very sad. Of Eight Chil-
dren now living, Five are Deaf and Dumb.
(And, I suppose, Dumb because Deaf).

You desire my Directions, how best to
supply that Defect: Having had some
Acquaintance (I understand) with Mr.
Alexander Popham, (who, I think, is yet
living) whom (being Born Deaf) I taught
(about Four or Five and Thirty Years ago)
to speak distinctly, (though I doubt he
may now have forgot much of it) and to
understand a Language, so as to express
his Mind (tolerably well) by Writing,
and to understand what is written to him
by others. As I had, before, taught Mr.
Daniel Whaley: Who was Deaf also;
but is lately dead.

Others, who were not Deaf, but had
great Impediments in their Speech, (who
Stutter'd extremely, or who have not been
able to pronounce some Letters,) I have
taught to Speak, very Distinctly, and to
Pronounce those Letters which before they
could not: So as perfectly to Conquer
that Difficulty; at least so as that it was
very little (if at all) discernable.

Some other Deaf Persons, I have not
attempted teaching them to Speak; but
only so as (in good Measure) to under-
stand a Language, and to express their
Mind (tolerably well) in Writing. Who
have thereby attained a much greater
Measure of Knowledge in many Things,
than was thought attainable to Persons in
their Circumstances; and become capable
(upon farther Improvement) of such fur-
ther Knowledge as is attainable by Read-
ing.

The former Part of this Work (teach-
ing to Speak or to speak Plain) is to be
done, by Directing them to Apply their
Tongue, Lips, and other Organs of Speech,
to such Postures and Motions, as are pro-
per for the Formation of such and such
Sounds (respectively) as are used in Speech.
And, then, the Breath, emitted from the
Lungs, will Form those Sounds: whether
the Person Speaking do hear himself, or
not.

Of which respective Formation, of all
Sounds commonly used in Speech, I have
given a full Account (and, I think, I am
the First who have done it) in my Trea-
tise De Loquela; prefixed to my Grammar
of the English Tongue; first Published in
the Year 1653. In Pursuance of which,
I attempted the Teaching of Deaf Persons
to speak.

And this is indeed the shorter Work
of the Two. (However looked up-
on the more Stupendous.) But this,
without the other, would be of little
Use. For, to pronounce Words only as a
Parrot, without knowing what they signi-
fie, would do us but little Service. And
it would by Degrees (without a Director
to correct Mistakes) come to be lost in
Part. For, like as one who Writes a fair
Hand, if he become Blind, would soon
forget the exact Draught of his Letters,
for want of an Eye to direct his Hand:
So he, who doth not Hear himself Speak,
must needs be apt to forget the Niceness of
Formation, (without a Prompter) for
want of an Ear to regulate his Tongue.

The other Part of the Work (to teach
a Language) is what you now inquire a-
bout.

In order to this; it is Necessary in the
First Place, That the Deaf Person be
taught to Write. That there may be some-
what to express to the Eye, what the
Sound (of Letters) represents to the Ear.

'Twill next be very Convenient, (be-
cause Pen and Ink is not always at Hand)
that he be taught, How to design each Let-
ter, by some certain Place, Position, or
Motion of a Finger, Hand, or other Part
of the Body, (which may serve instead
of Writing.) As for Instance, The Five
Vowels a e i o u; by pointing to the Top
of the Five Fingers: And the other Let-
ters be d c d, &c. by such other Place or Po-
sture of a Finger, or otherwise, as shall be,
agreed upon.

After this; a Language is to be taught
this Deaf Person, by like Methods as Chil-
dren are at first taught a Language;, (tho'
the Thing perhaps be not heeded.) Only
with this Difference: Children learn
Sounds by the Ear; but the Deaf Person
is to learn Marks (of those Sounds) by,
the Eye. But both the one and the other,
do equally signifie the same Things or No-
tions; and are equally (significantia ad
placitum) of meer Arbitrary Significati-
on.

'Tis then most natural (as Children
learn the Names of Things) to furnish
him (by Degrees) with a Nomenclator;
containing a competent Number of Names
of Things common and obvious to the Eye;
(that you may shew the Thing answering
to such a Name.) And these digested un-
der convenient Titles; and placed (under
them) in such convenient Order, (in se-
veral Columnes, or other orderly Situati-
on in the Paper) as (by their Position)
best to express, to the Eye, their Relation
or Respect to one another. As Contraries
or Correlatives, one over against the other;
Subordinates of Appurtenances, under their
Principals. Which may serve as a kind of
Local Memory.

Thus, (in one Paper) under the Title
Mankind, may he placed (not Confused-
ly, but in Decent Order) Man, Woman,
Child, (boy, girl.) And, if you please,
the Names of some known Persons, (of
the Family, or others,) with Spaces left
to be supplied with other like Names or
Words, as after there may be Occasi-
on.

Then (in another Paper) under the Ti-
tle Body may be written (in like conve-
nient Order) head, (hair, skin, ear,)
Face, forehead, eye, (eye-lid, eyebrow,)
cheek, nose, (nostril,) mouth, (lip, chin.)
Neck, (throat.) Back, Breast, Side,
(right-side, left-side.) Belly, Shoulder,
Arm, (elbow, wrist, hand, (back, palm,))
finger, (thumb, knuckle, nail.) Thigh,
knee, leg, (shin, calf, anckle,) foot, (heel,
sole,) toe. With like Spaces, as before,
for more to be added, as there is Occasi-
on.

And when he hath learned the Import
of Words in each Paper, let him write
them (in like manner) in distinct Leaves
or Pages of a Book (prepared for that
purpose) to confirm his Memory, and to
have Recourse to it upon Occasion.

In a Third Paper, you may give him
the Inward Parts. As, Scul, (brain,)
throat, (windpipe, gullet,) stomach,
(guts,) heart, lungs, liver, splene, kidney,
bladder, (urine,) vein, (blood,) bone,
(marrow,) flesh, fat, &c.

In another Paper, under the Title Beast,
may be placed; Horse, (stone horse, geld-
ing,) mare, (colt.) Bull, (ox,) cow,
calf. Sheep, ram, (wether,) ew, (lamb.)
Hog, boar, sow, pig. Dog, (mastiff,
hound, grey-hound, spaniel) bitch, (whelp,
puppy.) Hare, rabbet. Cat, mouse,
rat, &c.

Under the Title Bird, or Fowl, put
Cock, (capon,) hen, chick. Goose, (gan-
der,) gosling. Duck, (drake,) Swan, Crow,
Kite, Lark, &c.

Under the Title Fish, put Pike, Eel,
Plaice, Salmon, Lobstar, Crab, Oister,
Crawfish, &c.

You may then put Plants or Vegetables
under several Heads, or Subdivisions of
the same Head. As, Tree, (root, body,
bark, bough, leaf, fruit,) Oak, ash, ap-
ple-tree, pear-tree, vine, &c. Fruit, ap-
ple, pear, plumb, cherry, grape, nut, o-
range, lemon. Flower; rose, tulip, gilo-
fer. Herb, (weed,) grass. Corn, wheat,
barly, rye, pea, bean.

And the like of Inanimates. As, Hea-
ven; sun, moon, star. Elements; earth,
water, air, fire. And (under the Title
Earth;) clay, sand, gravel, stone. Me-
tal; gold, silver, brass, (copper,) iron,
(steel,) lead, tin, (pewter,) glass. Un-
der the Title Water; put Sea, pond, ri-
ver, stream. Under that of Air; put
Light, dark, mist, fog. Cloud; wind,
rain, hail, snow; thunder, lightning,
rainbow. Under that of Fire; Coal,
flame, smoak, soot, ashes,

Under the Title Clothes; put Woollen,
(cloth, stuff,) Linnen; (Holland, lawn
lockarum) Silk, (Satin, Velvet.) Hat,
cap, band, doublet, breeches, coat, cloak,
flocking, shooe , boot, shirt, petticoat,
gown, &c.

Under the Title' House; put Wall,
roof, door, window, (casement,) room.

Under Room; put Shop, hall, parlour,
dining-room, chamber, (study, closet,)
kitchin, cellar, stable, &c.

And, under each of these, (as distinct
Heads,) the Furniture or Utensils belong-
ing thereunto; (with Divisions and Sub-
divisions, as there is Occasion;) which I
forbear to mention, that I be not too pro-
lix.

And, in like manner, from time to
time, may be added more Collections or
Classes of Names or Words, conveniently
digested under distinct Heads, and suit-
ble Distributions; to be written in di-
stinct Leaves or Pages of his Book; in such
Order as may seem convenient: Which I
leave to the Prudence of the Teacher.

When he is furnished with a competent
Number of Names, (though not so many
as I have mentioned:) it will be seasona-
ble to teach him (under the Titles Singu-
lar, Plural,) the Formation of Plurals
from Singulars; by adding s or es. As,
Hand, hands; Face, faces; Fish, fishes;
&c. with some few Irregulars; As, Man,
Men; Woman, women; Foot, feet; Tooth,
teeth; Mouse, mice; Lowse, lice.; Ox,
Oxen, &c.

Which (except the Irregulars) will
serve for Possessives (to be after taught
him,) which are formed from their Pri-
mitives, by like Addition of s or es. Ex-
cept some few Irregulars; As My, mine;
Thy, thine; Our, ours; Your, yours; His,
her, hers; Their, theirs, &c.

And in all those, and other like Cases,
it will be proper first to shew him the
Particulars, and then the General Ti-
tle.

Then teach him (in another Page, or
Paper) the Particles; A, an; The, this,
that; These, those.

And the Pronouns; I, me, my, mine;
Thou, thee, thy, thine; We, us, our, ours;
Ye, you, your, yours; He, him, his; She,
her, hers; It, its; They, them, their,
theirs; Who, whom, who's.

Then, under the Titles Adjective, Sub-
stantive; teach him to connect these. As,
My hand, Your head, His foot, His feet,
Her arm, arms, Our hats, Their shoes,
John's coat, William's hand, &c

And, in order to furnish him with more
Adjectives; Under the Title Colours, you
may place Black, white, gray, green, blue,
yellow, red, &c. And, having shewed the
Particulars, let him know, These are cal-
led Colours.

The like for Taste, and Smell; As,
Sweet, bitter, soure, stink.

And for Hearing; Sound, noise, word.

Then, for Touch or Feeling; Hot,
(warm,) Cold, (cool,) Wet, (moist,)
Dry; Hard, soft; Tough, brittle; Hea-
vy, light, &c.

From whence you may furnish him
with more Examples of Adjectives with
Substantives; As, White bread, Brown
bread, Green grass, Soft cheese, Hard
cheese, Black hat, my black bat, &c.

And then, inverting the Order, Sub-
stantive and Adjective (with the Verb Co-
pulative between:) As; Silver is white;
Gold is yellow; Lead is heavy; Wood is
light; Snow is white; Ink is black; Flesh
is soft; Bone is hard; I am sick; I am
not well, &c. Which will begin to give
him some Notion of Syntax.

In like manner, when Substantive and
Substantive are so connected. As; Gold is
a Metal; A Rose is a Flower; We are
Men; They are Womn; Horses are
Beasts; Geese are Fowl; Larks are
Birds, &c.

Then, as those before relate to Quality,
you may give him some other Words rela-
ting to Quantity. As, Long, short; Broad,
narrow; Thick, thin; High (tall,) low,
Deep, shallow; Great, (big,) small, (li ti-
tle;) Much, little; Many, few; Full,
empty; Whole, part, (piece;) All, some,
none; Strong, weak; Quick, slow; Equal,
unequal; Bigger, less.

Then, Words of Figure; As, Streight,
crooked; Plain, bowed; Concave, (hollow)
convex; Round, square, three-square;
Sphere, (globe, ball, boul) Cube, (die,)
Upright, sloping; Leaning forward, lean-
ing backward; Like, unlike.

Of Gesture; As, Stand, lye, sit, kneel,
stoop.

Of Motion; As, Move, (stir,) rest;
Walk, (go, come;) Run; Leap; Ride;
Fall, rise; Swim, sink, (drown;) Slide;
Creep, (crawl;) Fly; Pull, (draw,) thrust,
throw; Bring, fetch, carry.

Then, Words relating to Time, Place,
Number, Weight, Measure, Money, &c.
are (in convenient time) to be shewed
him distinctly. For which the Teacher,
according to his Discretion, may-take a
convenient Season.

As likewise, the Time of the Day; The
Days of the Week; The Days of the Month;
The Months of the Year; and other things
relating to the Almanack. Which he will
quickly be capable to understand, if once
Methodically shewed him.

As likewise, the Names and Situations,
of Places, and Countries, which are con-
venient for him to know. Which may
be orderly written in his Book; and shew-
ed him in Maps of London, England, Eu-
rope, the World, &c.

But these may he done at leisure; As
likewise the Practice of Arithmetick, and
other like pieces oi Learning.

In the mean Time, (after the Concord
of Substantive and Adjective,) he is to be
shewed (by convenient Examples) that
of the Nominative and Verb. As for In-
stance, I go, You see, He sits, They stand,
the Fire burns, the Sun shines, the Wind
blows, the Rain falls, the Water runs;
and the like: with the Titles in the Top,
Nominative, Verb.

After this (under the Titles, Nomina-
tive, Verb, Accusative,) give him Exam-
ples of Verbs Transitives; As, I see you,
You see Me, The Fire burns the Wood, The
Boy makes a Fire, The Cook roasts the
Meat, The Butler lays the Cloth, We eat
our Dinner.

Or even with a Double Accusative; as
You teach me (Writing, or) to write;
John teacheth me to Dance; Thomas tells
me a Tale, &c.

After this; you may teach him the
Flexion or Conjugation of a Verb; or wbat
is equivalent thereunto. For, in our Eng-
lish Tongue, each Verb hath but Two Ten-
ses, (the Present and the Preter) and
Two Participles, (the Active and the Pas-
sive.) All the rest is performed by Au-
xiliaries. Which (Auxiliaries) have no
more Tenses, than the other Verbs.

Those Auxiliaries are, Do, did; Will,
would; Shall, should; May, might; Can,
could; Must, ought to; Have, had; Am,
(be,) Was. And if, by Examples, you
can insinuate the signification of these Few
Words: you have taught him the whole
Flexion of the Verb.

And here it will be convenient, (once
for all,) to write him out a full Para-
digm of some one Verb, (suppose, to See,)
through all those Auxiliaries.

The Verb it self, hath but these Four
Words to be learned; See, saw, seeing,
seen. Save that, after Thou in the Second
Person singular (in both Tenses) we add
est; and in the Third Person singular (in
the Present Tense) eth or es: Or, in-
stead thereof, st, th, s. And so in all
Verbs.

Then, to the Auxiliaries, Do did, Will
would, Shall should, May might, Can could,
Must, ought to, we adjoin the Indefinite
See. And, after Have had, Am (be) was,
the Passive Participle Seen. And so for
all other Verbs.

But the Auxiliary Am or Be, is some-
what Irregular; in a double Form;

Am, art, is; Plural, Are. Was, wast,
was; Plural, Were.

Be, beest, be; Plural, Be. Were, wert,
were; Plural, Were.

Be (am,) was, being, been.

Which (attended with the other Aux-
iliaries) make up the whole Passive Voice.

All Verbs (without Exception) in the
Active Participle, are formed by adding
ing; As, See, seeing; Teach, Teaching, &c.

The Preter Tense, and the Passive Par-
ticiple, are formed (regularly) by adding
ed. But are oft subject to Contractions,
and other Irregularities, (sometime, the
same in both; sometime, different.) And
therefore it is convenient, here, to give a
Table of Verbs (especially the most usu-
al) for those Three Cases. (Which
may, at once, teach their Signification,
and Formation,) As, Boil, boiled, boiled;
Rost, rosted, rosted; Bake, baked, baked,
&c. Teach, taught, taught; Bring, brought,
brought; Buy, bought, bought, &c. See,
saw, seen; Give, gave, given; Take, took,
taken; Forsake, forsook, forsaken; Writ,
wrote, written, &c. With many more, fit
to be learned.

The Verbs being thus dispatched; he is
then to learn the Prepositions. Wherein
lies the whole Regimen of the Noun.
(For Diversity of Cases we have none.)
The force of which is to be insinuated by
convenient Examples, suited to their diffe-
rent Significations. As, for instance,

Of, A piece of bread; A pint of Wine;
The cover of a pot; The colour of gold;
A ring of gold; A cup of Silver; the
Mayor of London; The longest of all, &c.

And in like manner for, Off; on, upon;
To, unto, till, until; From; At; In
(within,) out (without;) Into, out of;
About, over, under; Above, below; Be-
tween, among; Before, behind, after; For,
By; With, through; against; Concerning;
And by this Time, he will be pretty well
inabled to understand a Single Sentence.

In the last Place; he is (in like man-
ner) to be taught Conjuncticns. (Which
serve to connect, not Words only, but, Sen-
tences.) As, And, also, Likewise; Either,
or, whether; Neither, nor; If, when;
Why, (Whefore,) because, therefore;
But, through, yet, &c. And these illustra-
ted by convenient Examples, in each Case,
As,

Because I am cold; therefore I go to
the Fire; that I may be warm; For it is
cold Weather.

If it were fair, then it would be good
walking: But (however) though it rain,
yet I must go; because I promised. With
other like Instances.

And, by this time, his Book, (if well
furnished with plenty of Words; and
those well digested, under several Heads,
tod in good Order; and well recruited,
from time to time, as new Words occur;)
will serve him In the Nature of a Dictio-
nary and Grammar.

And, in Case the Deaf Person be other-
wise of a good Natural Capacity; and
the Teacher of a good Sagacity; By this
Method (proceeding gradually, step by
step,) you may (with Diligence and due
Application, of Teacher and Learner,) in
a Year's Time, or thereabouts, perceive a
greater Progress than you would expect:
and a good Foundation laid for further In-
struction, in Matters of Religion, and other
Knowledge which may be taught by Books.

It will be convenient, all along, to have
Pen, Ink and Paper ready at Hand, to write
down in Words, what you signifie to him
by Signs; and cause Him to write, (or
shew him how to write) what He signi-
fies by Signs. Which way (of signify-
ing their Mind by Signs) Deaf persons
are often very good at. And we must en-
deavour to learn their Language, (if I
may so call it) in order to teach them
ours: By shewing what Words answer
to their Signs.

'Twill be convenient also, as you go a-
long, (after some convenient progress
made) to express (in as plain Language
as may be) the import of some of the
Tables. As, for instance,

The Head is the Highest part of the
Body; the Feet, the Lowest part; The
Face is the Fore-part of the Head; The
Forehead is Over the Eyes; The Cheeks are
Under the Eyes; The Nose is Between the
Cheeks; The Mouth is Under the Nose,
and Above the Chin, &c.

And such plain Discourse, put into
Writing, and particularly explain'd; will
teach him by Degrees to understand Plain
Sentences. And like Advantages, a Saga-
cious Teacher, may take as Occasion of-
fers it self from time to time.

Thus I have, in a long Letter, given
you a Short Account of my Methods,
(used, in such Cases, with good Success)
which to do at Large, would require a
Book.

I have taken the pains to draw up this
Method, (which is what I have pursued
my self in the like Case,) as appprehend-
ing it may be of use to some others when I
am dead. And I am not desirous it should
die with me.

And I have done it as plainly as I could
that it may be the better understood.

I have given only some short Specimens
of such Tables as I had made for my own
Use, and the Use of those whom I was to
instruct; but to give them at large, would
be more than the Work of a Letter; and
they are to be varied, as the Circumstan-
ces of the Persons, and the Places may re-
quire, or the Prudence of a Teacher shall
find expedient.

It is adjusted to the English Tongue, be-
cause such were the Persons I had to deal
with.

To those of another Language, it must
be so altered as such Language requires.
And perhaps will not be so easily done
for another Language as for the Eng-
UJh. The Flexion of Nouns, the Conjuga-
tion of Verbs, the Difference of Genders,
the Variety of Syntax, &c. doth in other
Languages give a great deal of Trouble,
which the Simplicity of our Language
doth free us from. But this is not my
present Business.

I Am

SIR,

Yours to Serve You,

JOHN WALLIS.


FINIS.
