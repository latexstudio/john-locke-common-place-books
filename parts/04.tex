A Letter of Doctor John Wallis to Ro-
bert Boyle, Esq; concerning the said
Doctors Essay of Teaching a Person
Dumb and Deaf to Speak, and to
Understand a Language; together
with the Success thereof, made appa-
rent to his Majesty, the Royal Society,
and the University of Oxford.

SIR,

I Did acquaint you a while since, That
(beside the Consideration of .......,
which I had in Hand) I had under-
taken another Task, (almost as Hard as
to make Mr. ....... understand Reason) to
Teach a Person Dumb and Deaf, to Speak,
and to Understand a Language. Of which if
he could do either, the other would be more
easie; but his knowing neither makes
both harder: And tho' the former may
be thought the more difficult, the latter
may perhaps require as much of Time.
For if a considerable Time be requisite,
for him that can speak One, to learn a Se-
cond Language, much more for him that
knows None, to learn the First.

I told you in my last, that my Mute was
now at least Semivocalis; whereof because
you desire a more particular Information,
I thought my self oblig'd to give you this
brief Account of that whole Affair, that
you may at once perceive, as well upon
what Considerations I was induced to At-
tempt that Work, and what I did propose
to my self as Feasible therein, as what
Success had hitherto attended that Essay.

The task t self consists of Two Very
different Parts, each of which doth render
the other more difficult. For, beside that
which appears upon the First View, to
teach a Person who cannot Hear to Pro-
nounce the Sound of Words; there is that
other, of teaching him to Understand a
Language, and know the Signification of
those Words whether Spoken or Written,
whereby he may both express his own
Sense, and understand the Thoughts of o-
thers without which latter, that former
were only to speak like a Parrot, or to
write; like a Scrivener, who understanding
no Language but English, transcribes a
Piece of Latin, Welsh, or Irish; or like a
Printer of Greek or Arabick, who knows
neither the Sound nor Signification of what
he Printeth.

Now though I did not apprehend ei-
ther of these impossible; yet, that each
of them doth render the other more hard,
was so obvious as that I could not be igno-
rant of it. For how easily the Understand-
ing of a Language is attain'd by the Bene-
fit of Discourse we see every Day; not
only in those, who knowing One
Language already, are now to learn a Se-
cond, but (which doth more resemble
the present Case) in Children, who as yet
knowing none, are now to learn their First
Language.

For it is very certain, that no Two Lan-
guages can be so much different the one
from the other, but that the Knowledge of
the one will be subservient to the Gaining
of the other; not only because there is
now a Common Language, wherein the
Teacher may Interpret to the Learner the
Signification of those Words and Notions
which he knows not, and express his own
Thoughts to him; but likewise (which
is very considerable) because the Common
Notions of Language , wherein all or most
Languages do agree, and also so many of
the Particularities thereof as are common
to the Language he knows already, and
that which he is to learn, (which will be
very many) are already known; and
therefore a very considerable part already
dispatch'd of that Work, which will be
necessary for the Teaching of a First Lan-
guage to him who as yet knows none.

But to this Disadvantage (of teaching
a First Language) when that of Deafness
is superadded it must needs augment the
Difficulty; since it is manifestly evident
from Experience, that the most Advantag-
geous Way of Teaching a Child his First
Language is that of Perpetual Discourse;
not only what is particularly address'd to
himself, as well in pleasing Divertise-
ments, or delightful Sportings, (and
therefore insinuates itself without any irk-
some or tedious Labour) as what is direct-
ly intended for his more serious Infor-
mation: But that Discourse also which
passeth between others, where, without
Pains or Study, he takes Notice of what
Actions in the Speaker do accompany such
Words, and what Effects they do produce
in those to whom they are directed; which
doth, by Degrees, insinuate the Intendments,
of those Words.

And as that Deafness makes it the more
difficult to teach him a Language, so on
the other Hand that Want of Language
makes it more hard to teach him how to
speak or pronounce the Sounds. For there
being no other Way to direct his Speech,
than by teaching him how the Tongue,
the Lips, the Palate, and other Organs of
Speech are to be apply'd and mov'd in the
Forming of such Sounds as are required;
to the End that he may, by Art, pronounce
those Sounds which others do by Custom,
they know not how. It may be thought
hard enough to express in Writing, even
to one who understands it very well, those
very Nice Curiosities and Delicacies of Mo-
tion, which must be observed (though
we heed it not) by him, who without
Help of his Ear to guide his Tongue, shall
form that Variety of Sounds we use in
Speaking: Many of which Curiosities
are so Nice and Delicate, and the Diffe-
rence in Forming those Sounds so very Sub-
tile, that most of our selves, who pro-
nounce them every Day, are not able,
without a very Serious Consideration, to
give an Account by what Art or Motion
our selves form them; much less to teach
another how it is to be done. And if,
by writing to one who understands a Lan-
guage, it be thus difficult to give Instru-
ctions, how, without the Help of Hear-
ing, he must utter those Sounds, it must
needs increase the Difficulty, when there
is no other Language to express it in, but
that of Dumb Signs.

These Difficulties (of which I was well
aware) did not yet so far discourage me
from that Undertaking, but that I did still
conceive it possible that both Parts of this
Task might be effected.

As to the First of them; Tho' I did not
doubt but that the Ear doth as much guide
the Tongue in Speaking, as the Eye doth
the Hand in Writing, or Playing on the
Lute; and therefore those who by Acci-
dent do wholly, lose their Hearing, lose al-
so their Speech, and consequently become
Dumb as well as Deaf, (for it is in a
manner the same Difficulty for one that
Hears not, to speak well, as for him that
is Blind, to write a fair Hand). Yet since
we see that 'tis possible for a Lady to at-
tain so great a Dexterity, as, in the Dark,
to play on a Lute, though to that Va-
riety of nimble Motions, the Eyes Dire-
ction, as well as the Judgment of the Ear,
might seem necessary to guide the Hand;
I did not think it impossible, but that the
Organs of Speech might be taught to ob-
serve their due Postures, though neither
the Eye behold their Motion, nor the Ear
discern the Sound they make.

And as to the other, that of Language,
might seem yet more possible: For since
that in Children, every Day the Know-
ledge of Words, with their various Con-
structions and Significations, is by degrees
attain'd by the Ear, so that in a few Years
they arrive to a competent Ability of ex-
pressing themselves in their First Language,
at least as to the more usual Parts and No-
tions of it, why should it be thought
impossible that the Eye (though with
some Disadvantage) might as well apply
such Complication of Letters, or other
Characters, to represent the various Con-
ceptions of the Mind, as the Ear, a like
Complication of Sounds? For though,
as things now are, it be very true that
Letters are, with us, the immediate Chara-
cters of Sounds, as those of Sounds are of
Conceptions, yet is there nothing, in the
Nature of the Thing it self, why Letters
and Characters might not as properly be
applied to represent immediately, as by
the Intervention of Sounds, what our Con-
ceptions are.

Which is so great a Truth, (though
not so generally taken Notice of) that
'tis practiced every Day; not only of the
Chineses, whose whole Language is said
to be made up of such Characters as to
represent Things and Notions indepen-
dent on the Sound of Words; and is there-
fore indifferently spoken by those who
differ not in the Writing of it; (like as
what, in Figures, we write 1, 2, 3, for
One, Two, Three; a Frenchman, for Ex-
ample, reads Un, Deux, Trois) But, in
Part, also amongst our selves; as in the
Numeral Figures now mentioned, and ma-
ny other Characters of Weights and Me-
tals, used indifferently by divers Nations
to signifie the same Conceptions, though
expressed by a different Sound of Words;
and more frequently in the Practice of
Specious Arithmetick, and Operations of
Algebra, expressed in such Symbols, as so
little need the intervention of Words to
make known their Meaning, that, when
different Persons come to express, in Words
the Sense of those Characters, they will as
little agree upon the same Words, tho'
all express the same Sense as Two Transla-
tors of one and the same Book into ano-
ther Language.

And though I will not dispute the Pra-
ctical Possibility of introducing, an Univer-
sal Character, in which all Nations, tho'
of different Speech, shall express their
common Conceptions; yet that some Two
or Three (or more) Persons may, by Consent,
agree upon such Characters, whereby to ex-
press each to other their Sense in Writing,
without attending the Sound of Words, is
so far from an Impossibility, that it must
needs be allowed to be very Feasible if
not Facile. And if it may be done by new
invented Characters, why not as well by
those already in use? Which though to
those who know their common Use may
signifie Sounds; yet to those who know it
not, or do not attend it, may be as imme-
diately applied to signifie Things or Noti-
ons, as if they signified nothing else; and
so long as it is purely Arbitrary by what
Character to express such a Thing or No-
tion, we may as well make use of that
Character or Collection of Letters, to ex-
press the Thing to the Eyes of him that is
Deaf, by which others express the Sound
or Name of it to those that Hear. So that
indeed that shall be to him a real Chara-
cter, which expresseth to another a Vocal
Sound, but signifieth to both the same
Conception; which is to understand the
Language.

To these Fundamental Grounds of Pos-
sibility in Nature, I may next add a Con-
sideration which made me think it Moral-
ly possible; that is, not impossible to suc-
ceed in Practice. And because I am now
giving an Account to One who is so good
a Friend to Mathematicks, and Profi-
cient therein, I shall not doubt but this
Consideration will have the Force of a
great Suasive. Considering therefore from
how few and despicable Principles the
whole Body of Geometry, by continual
Consequence, is inforced; if so fair a Pile,
and curious Structure, may be rais'd, and
stand fast upon so small a Bottom, I could
not think it incredible, that we might at-
tain some considerable Success in this De-
sign, how little soever we had first to be-
gin upon; and from those little Actions
and Gestures, which have a kind of Na-
tural Significancy in them, we might, if
well managed, proceed gradually to the
Explication of a Compleat Language, and
withal direct to those Curiosities of Moti-
on and Posture in the Organs of Speech,
requisite to the Formation of a Sound de-
sired, and, so to effect both Parts of what
we intend.

My next Inducement to undertake it,
was a Consideration of the Person (which,
in a Work of this Nature, is of no small
Concernment) who was represented to
me as very Ingenious and Apprehensive,
(and therefore a very fit Subject to make
an Essay upon) and so far at least a Ma-
thematician as to draw Pictures; wherein,
I was told he had attain'd so good Ability,
which did induce me to believe that he
was not uncapable of the Patience, which
will be necessary to attend the Curiosity of
those little Varieties in the Articulation of
Sounds, being already accustomed to ob-
serve and imitate those little Niceties in
a Face, without which it is not possible to
Draw a Picture well.

I shall add this also, That, once, he could
have spoken, though so long ago that (I
think) he doth scarce remember it. But
having, by Accident, when about Five
Years of Age, lost his Hearing, he conse-
quently lost his Speech also; not all at
once, but by degrees, in about half a
Year's Time: Which though it do con-
firm what I was saying but now, how need-
ful it is for the Ear to guide the Tongue
in Speaking, (since that Habit of Speak-
ing, which was attain'd by Hearing, was
lost with it) and might therefore discou-
rage the Understanding; yet I was there-
by very much secured, that his Want of
Speech was but a Consequent of his Want
of Hearing, and did not proceed original-
ly from an Indisposition in the Organs of
Speech to form those Sounds. And tho'
the Neglect of it in his younger Years,
when the Organs of Speech being yet ten-
der, were more pliable, might now render
them less Capable of that Accurateness
which those of Children attain unto,
(whereof we have daily Experience, it
being found very difficult, if not impossi-
ble, to teach a Foreigner well in Years the
Accurate Pronouncing of that Sound of
Language, which, in his tender Years, he
had not learned) yet if he can attain to
speak but so well, as a Foreigner, at his
Years, may learn to speak English; what
shall be farther wanting to that Accurate-
ness which a Native from his Childhood
attains unto, may, to an indifferent Esti-
mate, be very well dispenced with?

Having thus acquainted you with those
Considerations which did induce me to at-
tempt it, lest you may think I build too
confidently there upon, and judge me guil-
ty of too much Vanity, in promising my
self a greater Success than can in Reason
be hoped for, it will next be necessary to
give you some Account what Measure of
Success I might propose to my self as pro-
bable in such an Undertaking.

And as to the First Part of it, (that of
Speaking) though I did believe, that much
more is to be Effected than is commonly
thought Feasible; and that it was possible
for him so to speak as to be understood;
yet I cannot promise my self that he shall
speak so Accurately, but that a Critical
Ear may easily discern some Failures, or
little Differences from the ordinary Tone
or Pronunciation of other Men; (since
that we see the like every Day, when not
Foreigners only, but those of our own
Nation in the remoter Parts of it, can
hardly speak so Accurately, as not to dis-
cover a considerable Difference from what
is the common Dialect or Tone at London.)
And this not only upon the Consideration
last mentioned, concerning the Organs, of
Speech less pliable to those Sounds to
which they were not from the First accu-
stomed) but especially upon that other
Consideration, concerning the Ears Useful-
ness to guide and correct the Tongue. For
as I doubt not but that a Person who knows
well how to Write, may attain by Custom
such a Dexterity as to Write in the Dark
tolerably well, yet it could not be expect-
ed that he should perform it with the same
Elegancy as if he saw the Motions of his
Hands; so neither is it reasonable to be
expected, that he who cannot Hear, tho'
he may know how to Speak truly, should
yet perform it so Accurately as if he had
the Advantage of his Ear also.

Nor can I promise, nor indeed hope,
that how Accurately soever he may learn
to Speak, he should be able to make so
great Use of it as others do. For since
that he cannot Hear what others say to
him, as well as express his own Thoughts
to them, he cannot make such Use of it
in Discourse as others may. And though
it may be thought possible that he may in
Time discern by the Motion of the Lips,
visible to the Eye, what is said to him,
(of which I am loth to deliver a positive
Judgment, since much may be said conje-
cturally both Ways), yet this cannot be
expected, till at least he be so perfectly
Master of the Language, as that, by a
few Letters known, he may be able to sup-
ply the rest of the Word; and by a few
Words, the rest of the Sentence, or at least
the Sense of it, by a probable Conjecture,
(as when we Decipher Letters written in
Cipher) For, that the Eye can actually
discern all the Varieties of Motion in the
Organs of Speech, and see what Sounds
are made by these Motions, (of which
many are inward, and are not expos'd to
the Eye at all) is not imaginable.

But as to the other Branch of our De-
sign, concerning the Understanding of a
Language, I see no Reason at all to doubt,
but that he may attain This, as perfectly as
those that Hear; and that, allowing the
like Time and Exercise, as to other Men
is requisite to attain the Perfection of a
Language, and the Elegance of it, he may
Understand as well, and Write as good
Language as other Men; and (abating
only what doth directly depend upon
Sound, as Tones, Cadencies, and such
Punctilio's) no whit inferior to what he may
attain to, if he had his Hearing as others
have. And what I speak of him in parti-
cular, I mean as well of any other Ingeni-
ous Person in his Condition; who, I be-
lieve, might be taught to use their Book
and Pen as well as others, if a right Course
were taken to that Purpose.

To tell you next, what Course I have
hitherto used towards this Design, it will
not be so necessary. For should I descend to
Particulars, it would be too tedious; es-
pecially since they are to be used very in-
differently, and varied as the present Case
and Circumstance do require; and as to
the General Way, it is sufficiently intima-
ted, already.

As to that of Speech, I must first, by
the most significant Signs I can, make him
to understand in what Posture and Motion
I would have him apply his Tongue, Lips,
and other Organs Of Speech, to the form-
ing of such a Sound as I direct. Which
if I hit right, I confirm him in it, if he
miss, I signifie to him in what he differed
from my Direction, and to what Circum-
stances he must attend to mend it. By
which Means, with some Trials and a
little Patience, he learns first One, then
another Sound; and, by frequent repeti-
tions, is confirm'd in it; or (if he chance
to forget) recovers it again.

And for this Work I was so far prepa-
red beforehand, that I had heretofore,
upon another Occasion, (in my Treatise
De Loquela, prefixed to my Grammar for
the English Tongue) considered very ex-
actly (what few Attend to) the Accurate
Formation of all Sounds in Speaking, (at
least as to our own Language, and those I
knew) without which it were in vain to
set upon the Task. For if we do not know,
or not consider, how we Apply our own Or-
gans in forming those Sounds we speak, it
is not likely, that we shall, this Way Teach
another.

As to that of Teaching him the Lan-
guage, I must, (as Mathematicians do
from a few Principles first granted) from
that little Stock (that we have to begin
upon) of such Actions and Gestures as have
a kind of Natural Significancy, or some
few Signs, which himself had before ta-
ken up to express his Thoughts as well as
he could, Proceed to Teach him what I
mean by somewhat else; and so, by Steps,
to more and more; And this, so far as
well I can, in such Methods, as that what
he knows already may be a Step to what
he is next to learn; as in Mathematicks,
we make use, not of Principles only, but
Propositions already demonstrated, in the
Demonstration of that which follows.

It remains now, for the Perfecting the
Account which at present you desire of me,
only to tell you, what progress we have al-
ready made; which had not your Desires
commanded from me, I should have re-
spited a while longer, till I might have
made it somewhat Fuller.

He hath been already with me somewhat
more than Two Months, in which Time,
though I cannot be thought to have Fi-
nished such a Work, yet the Success is not
so little as to Discourage the Undertaking,
but as much as I could hope for in so short
a Time, and more than I did expect; So
that I may say, the greatest Difficulty of
both Parts being almost over, what Re-
mains, is little more than the Work of
Time and Exercise. There is hardly any
Word, which (with Deliberation) he can-
not speak; but to do it Accurately, and
with Expedition, we must allow him the
Practice of some considerable Time, to
make it familiar to him.

And, as to the Language, though it were
very indifferent to him, who knew none,
which to begin withal; yet since it is out
of Question, that English, to him, is like
to be the most Useful and Necessary, it
was not adviseable to begin with any other.
For though he can pronounce the Latin
with much more Ease, (as being less per-
plexed with a Multitude of concurring
Consonants) yet this is a Consideration of
much less Concernment than the other.

To this therefore having apply'd him-
self, he hath already Learned a great ma-
ny Words, and, I may say, a considerable
Part of the English as to Words of most
frequent Use: But the whole Language
being so Copious, tho' otherwise Easie,
will require a longer Time to perfect what
he hath begun.

And this, Sir, is the full History of our
Progress hitherto. If you shall hereafter
esteem our future Success worthy your ta-
king notice of, you may command that, or
what else is within the Power, of

SIR, Your Honour's
very Humble Servant,
JOHN WALLIS.

Oxford,
March 14.
1664
