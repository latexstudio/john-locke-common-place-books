\documentclass[10pt, a6paper]{book}
\usepackage{extsizes}
\usepackage[veryoldstyle]{kpfonts}
\usepackage{fwlw}
\usepackage{yfonts}

\newcommand\VV{V\kern-0.3ex V}

\usepackage[a6paper, total={2.5in, 4in}, marginparwidth=0.7in, marginparsep=1ex]{geometry}
\addtolength{\textheight}{0.5in}
\usepackage{scalefnt}

\usepackage{tikz}
\usetikzlibrary{calc}
\usetikzlibrary{backgrounds}

\usepackage{fancyhdr}
\fancypagestyle{doublerule}{%
    \fancyhead{}
    \renewcommand{\headrulewidth}{1pt}
    \renewcommand{\headrule}{%
        \hbox to\headwidth{\leaders\hrule height \headrulewidth\hfill}%
        \hbox to\headwidth{\leaders\hrule height \headrulewidth\hfill}
    }
    \fancyfoot{}
    \fancyfoot[R]{\raisebox{0.8\thefontsize}{\usebox\NextWordBox}}
}
\makeatletter
\newlength{\thefontsize}
\setlength{\thefontsize}{\f@size pt}
\makeatother
\fancypagestyle{default}{%
    \renewcommand{\headrulewidth}{0pt}
    \fancyfoot[C]{\fontsize{\thefontsize}{\thefontsize}\selectfont\large\thepage}
    \fancyfoot[R]{
        \begin{tikzpicture}[overlay, black, every node/.style={draw=none}]
            \node[align=right, inner sep = 0pt] (next word)
                [anchor=north east] at (0, 2.5\thefontsize)
                {\usebox\NextWordBox};
        \end{tikzpicture}
    }
}
\fancyhf{}
\pagestyle{default}
\setlength{\headsep}{0.5\thefontsize}

\usepackage[side, symbol, flushmargin, perpage, ragged]{footmisc}
\renewcommand{\thefootnote}{\normalsize\raisebox{-0.5\baselineskip}{\fnsymbol{footnote}}}

\newlength{\gap}
\setlength{\gap}{0.02\textwidth}
\newlength{\hapage}
\setlength{\hapage}{0.5\textwidth}
\newlength{\shapeindent}
\setlength{\shapeindent}{0.065\textwidth}
\newlength{\shapelength}
\setlength{\shapelength}{\textwidth}
\addtolength{\shapelength}{-\shapeindent}

\title{A New Method of Making Common-Place-Books}
\author{John Locke}

\begin{document}\sloppy
\pagenumbering{gobble}
\thispagestyle{empty}
\begin{titlepage}
\begin{tikzpicture}[remember picture, overlay, every node/.style={draw=none}]
    \coordinate(center) at (current page.center);

    \newlength\framewidth
    \setlength\framewidth{0.65\paperwidth}
    \newlength\frameheight
    \setlength\frameheight{0.87\paperheight}
    \newlength\framegap
    \setlength\framegap{8pt}
    \coordinate(frame north) at ([yshift=0.5\frameheight]center);
    \coordinate(frame nw) at ([xshift=-0.5\framewidth]frame north);
    \coordinate(frame ne) at ([xshift=0.5\framewidth]frame north);
    \coordinate(frame south) at ([yshift=-0.5\frameheight]center);
    \coordinate(frame sw) at ([xshift=-0.5\framewidth]frame south);
    \coordinate(frame se) at ([xshift=0.5\framewidth]frame south);

    \newlength\titleskip
    \setlength\titleskip{0.2\thefontsize}
    \node[align=center] (A) [anchor=north] at ([yshift=-0.01\paperheight]frame north) {
        \Large A
    };
    \node[align=center] (new method) [anchor=north] at ([yshift=-\titleskip]A.south) {
        \LARGE N\,E\,W\, M\,E\,T\,H\,O\,D
    };
    \node[align=center] (of making) [anchor=north] at ([yshift=-\titleskip]new method.south) {
        \large OF MAKING
    };
    \node[align=center] (common place books) [anchor=north] at ([yshift=-\titleskip]of making.south) {
        \Large Common-Place-Books=~;
    };

    \node[align=center] (written) [anchor=north] at ([yshift=-\titleskip]common place books.south) {
        \large WRITTEN
    };
    \node[align=center] (author) [anchor=north] at ([yshift=-\titleskip]written.south) {
        \begin{minipage}{0.95\framewidth}
            \large
            \hangindent=2ex
            \noindent
            By the late Learned Mr.\ \emph{John Lock},
            Author of the \emph{ESSAY} \emph{concerning}
            \emph{Humane Understanding}.
        \end{minipage}
    };

    \node[align=center] (translated) [anchor=north] at ([yshift=-\titleskip]author.south) {
        \textgoth{\large Translated from the French.}
    };
    \draw[line width = 0.5pt]
        ([xshift=-0.475\framewidth]translated.north) -- ([xshift=0.475\framewidth]translated.north);
    \draw[line width = 0.5pt]
        ([xshift=-0.475\framewidth]translated.south) -- ([xshift=0.475\framewidth]translated.south);

    \node[align=center] (to which) [anchor=north] at ([yshift=-2\titleskip]translated.south) {
        T\,O\, W\,H\,I\,C\,H
    };
    \node[align=center] (le clerc) [anchor=north] at (to which.south) {
        \begin{minipage}{0.95\framewidth}
            \hangindent=2ex
            \noindent
            Is= added Something from Monsieur \emph{Le Clerc},
            relating to the same Subject.
        \end{minipage}
    };
    \node[align=center] (description) [anchor=north] at ([yshift=-\titleskip]le clerc.south) {
        \begin{minipage}{0.95\framewidth}
            \hangindent=2ex
            \noindent
            A TREATISE necessary for all Gentlemen,
            especially \emph{Students} of \emph{Divinity}, \emph{Physick}, and \emph{Law}.
        \end{minipage}
    };
    \node[align=center] (letters) [anchor=north] at ([yshift=-\titleskip]description.south) {
        \begin{minipage}{0.95\framewidth}
            \footnotesize
            \hangindent=2ex
            \noindent
            There are also added Two Letters=,
            containing a most Useful Method for instructing Persons= that are Deaf and Dumb,
            or that Labour under any Impediments= of Speech,
            to speak distinctly;
            writ by the late Learned Dr. \emph{John Wallis},
            Geometry Pr\oe fess. \emph{Oxon}, and \emph{F.R.S.}
        \end{minipage}
    };

    \node[align=center, text width = 0.95\framewidth] (printer) [anchor=south] at ([yshift=\titleskip]frame south) {
        \small
        Printed for \emph{J. Greenwood}, Bookseller,
        at the End of \emph{Cornhil}, next \emph{Stocks=-Market},
        1706.
    };
    \node[align=center] (location) [anchor=south] at (printer.north) {
        \emph{L\;O\;N\;D\;O\;N}:
    };

    \coordinate (terminator) at ($(letters.south)!0.5!(location.north)$);
    \draw[line width = 1pt]
        ([xshift=-0.475\framewidth]terminator) -- ([xshift=0.475\framewidth]terminator);

    \draw[line width = 0.5pt]
        (frame nw) rectangle (frame se);
    \draw[line width = 1pt]
        ($(frame nw) + (-\framegap, \framegap)$)
        rectangle
        ($(frame se) + (\framegap, -\framegap)$);
\end{tikzpicture}
\end{titlepage}
\cleardoublepage

\pagenumbering{arabic}
\fancyhead[C]{\fontsize{\thefontsize}{\thefontsize}\selectfont\Large\itshape Epistle Dedicatory.}
\fancyfoot[C]{}
\thispagestyle{doublerule}
\vspace*{\parskip}

\begin{center}
{
    \LARGE
    T\,O\\[0.5\baselineskip]
}
{
    \LARGE
    \emph{Mr.}\; Edward\; Northey,\\[0.5\baselineskip]
}
{
    \Large
    O\,F\\[\baselineskip]
}

{
    \Huge
    \textit{H\,A\,C\,K\,N\,E\,Y}.\\[\baselineskip]
}
\end{center}

{
    \LARGE
    S\,I\,R,
}

\newsavebox{\dropcap}
\savebox{\dropcap}{\Huge I}
\newlength{\dropcapheight}
\newlength{\dropcapwidth}
\settoheight{\dropcapheight}{\usebox{\dropcap}}
\settowidth{\dropcapwidth}{\usebox{\dropcap}}
\null\smash{
    \raisebox
        {-\dimexpr \dropcapheight+0.6\baselineskip\relax}
        {\noindent\usebox{\dropcap}}
}

\setlength{\shapeindent}{\dimexpr \dropcapwidth+1em\relax}
\setlength{\shapelength}{\textwidth}
\addtolength{\shapelength}{-\shapeindent}
\parshape=3
\shapeindent \shapelength
\shapeindent \shapelength
0pt \textwidth
\noindent
Here Present you with a \emph{Method} of making \emph{Common-Places=},
for which I need make no Apology,
when I shall have told you that it was= writ by that Great Master of Reason and Method,
the late Learned Mr.\ \emph{Lock}.

I know very well that nothing of that Great Man's= can fail of meeting with a kind Reception from you,
who have so often expressed the Pleasure and Advantage you receive from his= \emph{Writings=};
and I make no doubt but this= \emph{Method} will have its= Share in your Esteem,
by being observed in the Future Course of your Studies=.

I shall forbear saying any Thing of the Usefulness= of \emph{Common-Places=} in General,
it being Foreign to my present Purpose;
neither is= it my Business= here to relate how favourably the Learned, both Ancient and Modern, have spoken of them, and with what Success= they have us'd 'em.

It will be abundantly sufficient towards= their Recommendation if I tell you that \emph{Tully} was= One among the Former, and Mr.\ \emph{Lock} among the Latter.

But I shall refer you to what I have extracted from the very Learned \emph{Monsieur le Clerc} concerning this= Matter.

It may be expected that I should give some Account of this= \emph{Method};
all that I shall say at present is=, That Mr.\ \emph{Lock} having drawn it up during his= Travels= abroad, communicated it to several of his= Friends=, who mighti ly importun'd him to make it Publick;
but he for a long Time declin'd it, (for Reasons= which you will find in his= \emph{Epistle} prefixed to this= Treatise) till at last, in Compliance with their repeated Requests=, he gave it to \emph{Monsieur le Clerc}, who in the Year 1686 Publish'd it in \emph{French}, in the Second Tome of the \emph{Bibliotheque Universelle}.

This= \emph{Method} having met with General Approbation from the Learned, I thought it a Pity that any Thing of Mr.\ \emph{Lock}'s= should be hid from any of his= Country-men, in an unknown Tongue:
I have therefore made it speak \emph{English}, and taken the Freedom of Dedicating it to you, with the Addition of Two Letters=, (because of their Publick Use) containing an extraordinary and most useful Method how to \emph{Teach} Deaf and Dumb Folks= to Speak and \VV rite a Language, invented by that Great Man \emph{John Wallis=}, Dr.\ in \emph{Divinity}, \emph{Geometry} Professor in \emph{Oxford}, and Fellow of the \emph{Royal Society}:
\VV ho, let it be Recorded to his= Immortal Honour, was= the First in \emph{England} that made \emph{Art} supply the Defects= of \emph{Nature}, in learning Persons= that were Deaf and Dumb to Speak and Write distinctly and intelligibly.
The Method that the Doctor prescribes= is= so Plain, Familiar and Demonstrative, that any Person of Common Ingenuity may attain this= Art with Ease, and abundance of Pleasure.

But I am afraid, Sir, I have been too tedious=, therefore I shall only add this=, May you go-on, as= you have already begun, to Cultivate a strict Friendship with Virtue and Learning;
and while many Young Gentlemen mind nought but the Gratifying their foolish Inclinations=, may you pursue the Ratio nal Pleasures= of the Mind, whose Fruits= are Solid Joy and Comfort;
in cited thereto on the one Hand by the good Example of your \VV orthy Parents=, as= on the other by that of your very Learned \emph{Uncle}, who so Gloriously Adorns= the Great \emph{Post} he is= in.

This=, Sir, is= the Hearty \VV ish and Desire of\\[\baselineskip]

{\LARGE\hfill Your Friend and most\hspace*{4ex}}\\[\baselineskip]

{\LARGE\hfill Humble Servant,\hspace*{4ex}}

\fancyhead[CE]{\fontsize{\thefontsize}{\thefontsize}\selectfont\Large Mr.\ \emph{Le Clerc}'s= Advice about}
\fancyhead[CO]{\fontsize{\thefontsize}{\thefontsize}\selectfont\Large the Use of \emph{Common-Places=}.}
\thispagestyle{doublerule}
\vspace*{0pt}
\setcounter{page}{1}
\renewcommand\thepage{\roman{page}}
\fancyhead[RO]{
    \begin{tikzpicture}[overlay, black, every node/.style={draw=none}]
        \node[align=right, inner sep = 0pt] (pagenumber)
            [anchor=south west] at (4ex, 0)
            {\thepage};
    \end{tikzpicture}
}
\fancyhead[LE]{
    \begin{tikzpicture}[overlay, black, every node/.style={draw=none}]
        \node[align=left, inner sep = 0pt] (pagenumber)
            [anchor=south east] at (-4ex, 0)
            {\thepage};
    \end{tikzpicture}
}
\fancyfoot[C]{}

\hangindent=2ex
{
    \Large
    \noindent
    Monsieur \emph{Le Clerc}'s= Character of Mr. \emph{LOCK}'S Method,\par
}
\begin{center}
{W\,I\,T\,H\, H\,I\,S}\\
\vspace{\baselineskip}
{\Huge A\,D\,V\,I\,C\,E}\\
\vspace{\baselineskip}
{About the}\\
\vspace{\baselineskip}
{\Huge U\;S\;E}\\
\vspace{\baselineskip}
{\large O\;F}\\
\vspace{\baselineskip}
{\LARGE Common-Places=.}
\end{center}

\savebox{\dropcap}{\scalefont{2.8}I}
\settoheight{\dropcapheight}{\usebox{\dropcap}}
\settowidth{\dropcapwidth}{\usebox{\dropcap}}
\smash{\raisebox%
    {-\dimexpr \dropcapheight+0.6\thefontsize\relax}%
    {\noindent\usebox{\dropcap}}}%

{\itshape
\setlength{\shapeindent}{\dimexpr \dropcapwidth+1em\relax}
\setlength{\shapelength}{\textwidth}
\addtolength{\shapelength}{-\shapeindent}
\parshape=3
\shapeindent \shapelength
\shapeindent \shapelength
0pt \textwidth
\noindent
N all Sorts= of Learning,
and especially in the Study of Languages=,
the Memory is= the \emph{Treasury or Store-house},
but the Judgment the \emph{Disposer},
which ranges= in Order whatever it hath drawn from the Memory:
But lest the Memory should be Oppressed,
or Over-burthen'd by too many Things=,
Order and Method are to be called in to its= Assistance.
So that when we extract any Thing out of an Author which is= like to be of future Use,
we may be able to find it without any Trouble.
For it would be to little Purpose to spend our Time in Reading of Books=,
if we could not apply what we read to our Use.
It would be just for all the World at serviceable as= a great deal of \emph{Houshold-Stuff},
when if we wanted any particular Thing we could not tell where to find it.
\footnote{Columella out of Cicero, L. 12. Cap. 11.}%
\emph{It is= an Old Saying,
    That that is= the Truest Poverty,
    when if you have Occasion for any Thing,
    you can't use it,
    because you know not where 'tis= laid.}
Many have Wrote much en this= Subject,
and I have made Trial of them,
but I have never met with a better and more easie Method,
than that which I receiv'd from a \footnote{He speaks= of this= Method of Mr. Lock's=.} Friend,
and publish'd in French some Time since.

And I have found, upon several Years= Experience,
this= Method, which is= very well adapted,
not only to the Latin, but also to the Greek Tongue,
to be extraordinary useful.
Neither do I ever look, upon my Latin or Greek Collections=,
but I call to mind the Kindness= of that Excellent and Learned Person,
who taught me that Method.

At the Entrance indeed upon any Study,
when the Judgment is= not sufficiently confirm'd,
nor the Stock of Knowledge over large,
so that the Student are not very well acquainted with what is= worth Collecting,
scarce any Thing is= Extracted,
but what will be useful but for a little while;
because as= the judgment grows= Ripe,
those Things= are despis='d which before were had in esteem.
Yet it is= of Service to have Collections= of this= Kind,
both that Students= may learn the Art of putting Things= in Order,
as= also the better retain what they Read.

But here are Two Things= carefully to be observed;
the First is=, that we extract only those Things= which are Choice and Excellent,
either for the Matter it self,
or else for the Elegancy of the Expression,
and not what comes= next;
for that Labour would abate our Desire to go on with our Reading;
neither are we to think that all those things= are to be writ out which are called
$\Gamma\nu\omega\mu\alpha\iota$, or Sentences=.
Those Things= alone are to be picked out,
which we cannot so readily call to mind,
or for which we should want proper Words= and Expressions=.

For Instance, although the Story in that Place of Virgil where these Words= are,

\emph{Discite Justitiam moniti, \& non temnere Divos=.}

\emph{Being warn'd, \emph{by all these Things=},
learn to do that which is= Just, and
not to despise the Gods=,}

is= worth taking Notice of,
yet I would not have you Write these Words= down,
because there is= Nothing in the Thing it self,
or in the Manner of Expression,
that is= above the Reach of any Ordinary Capacity.

The Second Thing which I would have taken Notice of, is=,
that you don't Write out too much,
but only what is= most Worthy of Observation,
and to mark the Place of the Author from whence you Extract it,
for otherwise it will cause the Loss= of too much Time.

Neither ought any Thing to be Collected whilst you are busied in Reading;
if by taking the Pen in Hand the Thread of your Reading be broken off,
for that will make the Reading both Tedious= and Unpleasant.

The Places= we design to extract from are to be marked upon a piece of Paper,
that we may do it after we have read the Book out;
neither is= it to be done just after the First Reading over of the Book,
but when we have read it a second time.

These Things= it's= likely may seem Minute and Trivial,
but without 'em great Things= cannot subsist;
and these being neglected cause very great Confusion both of Memory and Judgment,
and that which above all Things= is= most to be valued,
Loss= of Time.

Some who otherwise were Men of most extraordinary Parts=,
by the Neglect of these things= have committed great Errors=,
which if they had been so happy as= to have avoided,
they would have been much more serviceable to the Learned World,
and so consequently to Mankind.

And in good Truth,
They who despise such Things=,
do it not so much from any greater share of Wit that they have than their Neighbours=,
as= from Want of Judgment;
whence it is= that they do not well understand how Useful Things= Order and Method are.
}

\end{document}
